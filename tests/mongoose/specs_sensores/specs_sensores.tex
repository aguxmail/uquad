\documentclass[spanish,12pt,a4paper,titlepage]{report}
\usepackage[utf8]{inputenc}
\usepackage{graphicx}
\usepackage{subfig}
\usepackage{float}
\usepackage{wrapfig}
\usepackage{multirow}
\usepackage{caption}
\usepackage[spanish]{babel}
\usepackage[dvips]{hyperref}
\usepackage{amssymb}
\usepackage{listings}
\usepackage{epsfig}
\usepackage{amsmath}
\usepackage{array}
\usepackage[table]{xcolor}
\usepackage{multirow}
%\usepackage[Sonny]{fncychap}
\usepackage[Lenny]{fncychap}
%\usepackage[Glenn]{fncychap}
%\usepackage[Conny]{fncychap}
%\usepackage[Rejne]{fncychap}
%\usepackage[Bjarne]{fncychap}
%\usepackage[Bjornstrup]{fncychap}

%\usepackage{subfiles}
%\usepackage{framed}

\setlength{\topmargin}{-1.5cm}
\setlength{\textheight}{25cm}
\setlength{\oddsidemargin}{0.3cm} 
\setlength{\textwidth}{15cm}
\setlength{\columnsep}{0cm}

\begin{document}

\chapter{Especificaciones de los sensores disponibles en la IMU}
\label{chap-specs-sensores}


\section{Introducción}
\label{sec:intro}

El objetivo de este informe es resumir las características de los sensores con lo que se cuenta en la Mongoose, la IMU\footnote{Inertial Measurement Unit.} que se usará en el cuadricópter:
\begin{itemize}
\item Acelerómetro de 3 ejes.
\item Giróscopo de 3 ejes.
\item Barómetro.
\item Magnetómetro de 3 ejes.
\item Sensor de temperatura.
\end{itemize}

\newpage
\section{Acelerómetro}
\label{sec:acc}

\begin{table}[H]
\begin{center}
\rowcolors{1}{gray!20}{}
\begin{tabular}{|p{3cm}|p{6.5cm}|}
\hline
Rango & $\pm$16g \\
\hline
Resolución & 10-13 bits (siempre 4mg/LSB) \\
\hline
Datos nuevos &  0.1 a 3200 Hz\\
\hline
Ruído XY & 0.75@100Hz - 3@3200Hz LSB-rms\\
\hline
Ruído Z & 1.1@100Hz - 4.5@3200Hz LSB-rms\\
\hline
Cross Axis & $\pm$ 1\% \\
\hline
\end{tabular}
\label{tab:acc}
\end{center}
\end{table}

\textbf{NOTAS}:
\begin{itemize}
\item Output data rate de 800Hz require I$^2$C a 400kHz, mayor frecuencia require SPI.
\item Ancho de banda = $Datos\_nuevos/2$
\end{itemize}

\section{Giróscopo}
\label{sec:gyro}

\begin{table}[H]
\begin{center}
\rowcolors{1}{gray!20}{}
\begin{tabular}{|p{3cm}|p{6.5cm}|}
\hline
Rango & $\pm$2000$^\circ/s$ \\
\hline
Resolución & 14.475 LSB/($^\circ/s$) \\
\hline
Datos nuevos &  3.9Hz a 8kHz\\
\hline
Ancho de banda & 256Hz \\
\hline
Cross Axis & $\pm$ 2\% \\
\hline
Ruído & 0.38 $^\circ/s$-rms \\
\hline
\end{tabular}
\label{tab:gyro}
\end{center}
\end{table}

\textbf{NOTAS}:
\begin{itemize}
\item Datos nuevos: La muestras pasan por un LPF digital de 256 a 5Hz, esto limita el ancho de banda.
\end{itemize}

\newpage
\section{Magnetómetro}
\label{sec:magnetometro}

\begin{table}[H]
\begin{center}
\rowcolors{1}{gray!20}{}
\begin{tabular}{|p{3cm}|p{6.5cm}|}
\hline
Rango & $\pm$8 Ga\\
\hline
Resolución &  5mGa@GN=2\\
\hline
Datos nuevos &  0.75 - 75Hz\\
\hline
Ancho de banda &  75/2Hz\\
\hline
Cross Axis & $\pm$0.2\% FS/Ga \\
\hline
Ruído & ? \\
\hline
\end{tabular}
\label{tab:magn}
\end{center}
\end{table}

\textbf{NOTAS}:
\begin{itemize}
\item El rango queda determinado por la ganancia, que se configura con 3 bits:
\begin{table}[H]
\begin{center}
\rowcolors{1}{gray!20}{}
\begin{tabular}{|c|c|c|c|c|c|c|c|c|}
\hline
\textbf{GN} & 0 & 1 & 2 & 3 & 4 & 5 & 6 & 7 \\
\textbf{Rango} (Ga)& $\pm$0.88 & $\pm$1.3 & $\pm$1.9 & $\pm$2.5 & $\pm$4.0 & $\pm$4.7 & $\pm$5.6 & $\pm$8.1 \\
\hline
\end{tabular}
\label{tab:magn-gain}
\end{center}
\end{table}
\item Se puede configurar para que el dato que muestre sea el promedio de hasta 8 muestras.
\end{itemize}

\section{Barómetro}
\label{sec:barometro}

\begin{table}[H]
\begin{center}
\rowcolors{1}{gray!20}{}
\begin{tabular}{|p{3cm}|p{6.5cm}|}
\hline
Rango & 300 a 1100 hPa (9000 a -500m)\\
\hline
Resolución &  1Pa\\
\hline
Precición. Abs. & typ/max $\pm$1.0/$\pm$3.0 hPa \\
\hline
Precición Rel. & $\pm$0.5 hPa \\
\hline
Datos nuevos &  typ/max: 3/4.5ms - 17/25ms\\
\hline
Ancho de banda &  333/40Hz\\
\hline
Ruído (hPa) &  0.06 - 0.03\\
\hline
Ruído (m) & 0.5 - 0.25 \\
\hline
\end{tabular}
\label{tab:barometro}
\end{center}
\end{table}

\textbf{NOTAS}:
\begin{itemize}
\item El rango, en altura, se refiere a la altura sobre el nivel del mar.
\item El modo de operación (cantidad de muestras promediadas) afecta:
  \begin{itemize}
  \item El tiempo de conversión.
  \item El ancho de banda.
  \item La resolución.
  \item El ruído.
  \end{itemize}
\item Es necesario hacer una medida de temperatura de vez en cuando (1Hz) para mejorar la lectura del barómetro.
\end{itemize}

\section{Temperaturmometrorómetro}
\label{sec:temp}

\begin{table}[H]
\begin{center}
\rowcolors{1}{gray!20}{}
\begin{tabular}{|p{3cm}|p{6.5cm}|}
\hline
Rango & \\
\hline
Resolución &  0.1 $^\circ$C\\
\hline
Precición Abs. & typ/max $\pm$1.0/$\pm$2.0 $^\circ$C\\
\hline
Datos nuevos &  typ/max: 3/4.5ms\\
\hline
Ruído & ? \\
\hline
\end{tabular}
\label{tab:temp}
\end{center}
\end{table}

\textbf{NOTAS}:
\begin{itemize}
\item El sensor de temperatura está incorporado al barómetro.
\end{itemize}

\end{document}