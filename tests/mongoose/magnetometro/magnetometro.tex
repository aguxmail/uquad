\documentclass[spanish,12pt,a4paper,titlepage]{report}
\usepackage[utf8]{inputenc}
\usepackage{graphicx}
\usepackage{subfig}
\usepackage{float}
\usepackage{wrapfig}
\usepackage{multirow}
\usepackage{caption}
\usepackage[spanish]{babel}
\usepackage[dvips]{hyperref}
\usepackage{amssymb}
\usepackage{listings}
\usepackage{epsfig}
\usepackage{amsmath}
\usepackage{array}
\usepackage[table]{xcolor}
\usepackage{multirow}
%\usepackage[Sonny]{fncychap}
\usepackage[Lenny]{fncychap}
%\usepackage[Glenn]{fncychap}
%\usepackage[Conny]{fncychap}
%\usepackage[Rejne]{fncychap}
%\usepackage[Bjarne]{fncychap}
%\usepackage[Bjornstrup]{fncychap}

%\usepackage{subfiles}
%\usepackage{framed}

\setlength{\topmargin}{-1.5cm}
\setlength{\textheight}{25cm}
\setlength{\oddsidemargin}{0.3cm} 
\setlength{\textwidth}{15cm}
\setlength{\columnsep}{0cm}

\begin{document}

\chapter{Magnetómetro}
\label{chap:magnetometro}

\section{Objetivos}
%\ref{} agregar refs
%TODO esta ref es trucha! cambiar por la posta!!

El objetivo de estas pruebas es comprender y caracterizar el magnetómetro de 3 ejes Honeywell HMC5583, incorporado para asistir en la determinación de la orientación absoluta del cuadricóptero.

\section{Materiales}
\label{sec:materiales}

\begin{itemize}
\item Mesa de madera de 1.5m de largo.
\item Laptop.
\item Hilo y clavos.
\item Mesa nivelable.
\item Cubo de lapacho
\item IMU ``Mongoose'' de CKDevices, con un HMC5583.
\item Foto satelital, con información sobre coordenadas.
\item Mesa nivelable, con una superficie que se pueda inclinar en ángulos conocidos.
\item Escuadra
\end{itemize}

\newpage
\section{Procedimiento}
\label{sec:procedimiento}

\begin{enumerate}
\item Utilizar un punto lejano, y conocido, para alinear la mesa de 1.5m a una dirección conocida:
  \begin{itemize}
  \item Colocar un clavo en cada extremo de la mesa, de manera que la punto del clavo quede apuntando hacia el cielo.
  \item Atar un hilo entre los dos clavos (esto se utilizará más adelante).
  \item Colocarse cerca de uno de los clavos, y usar el otro para alinearse a un objeto conocido y distante, que sea posible de ubicar en una foto satelital.
%TODO cual se elijió?
  \end{itemize}

\end{enumerate}


\end{document}