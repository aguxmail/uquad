\documentclass[main]{subfiles}
\begin{document}

\chapter{Introducci�n}

La presente monograf�a presenta el trabajo llevado a cabo y los resultados obtenidos a lo largo del desarrollo del  proyecto de fin de carrera ``uQuad!''. El objetivo del mismo es dise�ar e implementar un veh�culo a�reo no tripulado (UAV, por sus siglas en ingl�s) a partir de un cuadric�ptero radiocontrolado.\\
\\
Para ello ser� necesario estudiar el comportamiento f�sico del sistema y modelar el mismo. En base a dicho modelo se podr� entonces dise�ar e integrar un sistema de control que permita el vuelo aut�nomo del sistema.\\
\\
El sistema necesitar� obtener informaci�n acerca de su estado y el de su entorno para poder llevar a cabo sus funciones. Dicha informaci�n ser� obtenida a trav�s de un conjunto de sensores, los cuales deber�n ser adaptados e integrados al sistema en forma adecuada. Se deber�, entonces, integrar la instrumentaci�n adecuada para obtener medidas de variables f�sicas fundamentales para realizar el control.\\
\\
A fin de definir el alcance del proyecto, debe aclararse que se entiende por vuelo aut�nomo la capacidad de mantener una posici�n y altitud determinada o la de seguir una ruta programable.El dispositivo deber� contar con algoritmos de control lo suficientemente robustos como para poder realizar maniobras simples en forma aut�noma en diversas condiciones, tanto en espacios interiores como exteriores.\\
\\
Si bien no se ha definido una aplicaci�n espec�fica para el sistema dise�ado, es claro que mediante modificaciones no demasiado grandes el mismo puede ser adaptado a m�ltiples usos.\\
\\
El desarrollo del proyecto implicar�, entonces:

\begin{itemize}
\item Analizar el dispositivo comercial con el fin de comprender el sistema a controlar
\item Analizar el comportamiento f�sico del sistema y modelar el mismo en forma adecuada
\item Realizar ingenier�a inversa al cuadric�ptero, su electr�nica y su sistema de control. En particular, se prestar� especial detalle a los protocolos utilizados para el control de motores.
\item Poner a punto la instrumentaci�n necesaria
\item Dise�ar el sistema de procesamiento de datos de los sensores
\item Implementar las vias de comunicaci�n necesarias para obtener datos a partir de los sensores, realizar el comando remoto del sistema, dar instrucciones e indicar la ruta a seguir al mismo, etc.
\item Dise�ar e integrar el sistema de control
\item Lograr que el cuadricoptero sea capaz de realizar el seguimiento de una ruta en forma aut�noma
\end{itemize}

\end{document}