\documentclass[main]{subfiles}
\begin{document}

\cleardoublepage
\phantomsection
\addcontentsline{toc}{part}{Introducci\'on}
\chapter*{Introducci\'on}


\pagenumbering{arabic}
Existe actualmente en el Instituto de Ingenier\'ia El\'ectrica de la Facultad de Ingenier\'ia de la Universidad de la Rep\'ublica una l\'inea de investigaci\'on orientada a la rob\'otica m\'ovil y al desarrollo de t\'ecnicas de control. En este marco, se han desarrollado diversas plataformas experimentales tanto por docentes del Departamento de Control como por estudiantes durante los proyectos de fin de carrera. En particular, dos proyectos de fin de carrera se orientaron al diseño de un UAV\footnote{Unmanned Aerial Vehicle} con arquitectura de avi\'on. El primero de ellos, \emph{AMFO1-Bobby}, se avoc\'o al diseño y construcci\'on de un avi\'on de dimensiones reducidas con capacidad de vuelo relativamente prolongada. Dicha plataforma fue dise\~nada para ser volada gracias a un control remoto. El proyecto AUION, continu\'o el trabajo resolviendo el diseño del controlador del avi\'on en la situac\'on de vuelo, quedando excluido el despegue y el aterrizaje del veh\'iculo.\\

En el presente proyecto se contin\'ua trabajando en la línea presentada, dise\~nando e integrando un sistema de control a un veh\'iculo a\'ereo comercial con arquitectura de cuadric\'optero de forma de lograr el vuelo aut\'onomo del mismo. Por vuelo aut\'onomo se entiende la capacidad de seguir una trayectoria autogenerada a partir de puntos del espacio introducidos por un usuario por los cuales se desea ``pasar'', de ahora en m\'as \emph{waypoints}. \\

La arquitectura escogida ofrece una gran variedad de aplicaciones entre las cuales se puede nombrar la fotograf\'ia y el cine, el relevamiento de zonas peligrosas o contaminadas, la  vigilancia y la navegaci\'on bajo techo. \\

No es el objetivo de este proyecto desarrollar un producto final, sino que por el contrario lo que se plantea es desarrollar una plataforma experimental sobre la cual otros grupos puedan trabajar, profundizando en diversas \'areas como la estimaci\'on de estados, t\'ecnicas de control m\'as avanzadas, la generaci\'on de rutas \'optimas y la cooperaci\'on entre varias plataformas id\'enticas.\\

El objetivo del presente proyecto incluye instrumentar dos cuadric\'opteros de forma de poder obtener todas las variables de estado con las que se elige representar al sistema. La comprensi\'on del comando de los motores y la caracterizaci\'on de los mismos son parte fundamental para el control del sistema. Se desarroll\'o una primer versi\'on de un generador de rutas que defina la trayectoria a seguir a partir de waypoints introducidos por el usuario. Se desarroll\'o un controlador que permita seguir la trayectoria generada por el generador de rutas. Se program\'o e integr\'o una CPU a cada cuadric\'optero para realizar las tareas hasta aqu\'i planteadas, manteniendo la posibilidad de trabajar en modo autom\'atico y manual.\\

Como ya fue explicitado, es de inter\'es que este proyecto sea la plataforma para el desarrollo de proyectos que permitan una profundizaci\'on en las l\'ineas de trabajo planteadas, por dicho motivo se incluye dentro del alcance el diseño de un simulador que permita, a lo largo de este y de futuros proyectos, testear los algor\'itmos de control en desarrollo.

\end{document}