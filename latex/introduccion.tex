\documentclass[main]{subfiles}
\begin{document}

\chapter*{Introducci\'on}
%TODO add contents line

La presente monograf\'ia presenta el trabajo llevado a cabo y los resultados obtenidos a lo largo del desarrollo del  proyecto de fin de carrera ``uQuad!''. El objetivo del mismo es dise\~nar e implementar un veh\'iculo a\'ereo no tripulado (UAV, por sus siglas en ingl\'es) a partir de un cuadric\'optero radiocontrolado.\\
\\
Para ello ser\'a necesario estudiar el comportamiento f\'isico del sistema y modelar el mismo. En base a dicho modelo se podr\'a entonces dise\~nar e integrar un sistema de control que permita el vuelo aut\'onomo del sistema.\\
\\
El sistema necesitar\'a obtener informaci\'on acerca de su estado y el de su entorno para poder llevar a cabo sus funciones. Dicha informaci\'on ser\'a obtenida a trav\'es de un conjunto de sensores, los cuales deber\'an ser adaptados e integrados al sistema en forma adecuada. Se deber\'a, entonces, integrar la instrumentaci\'on adecuada para obtener medidas de variables f\'isicas fundamentales para realizar el control.\\
\\
A fin de definir el alcance del proyecto, debe aclararse que se entiende por vuelo aut\'onomo la capacidad de mantener una posici\'on y altitud determinada o la de seguir una ruta programable.El dispositivo deber\'a contar con algoritmos de control lo suficientemente robustos como para poder realizar maniobras simples en forma aut\'onoma en diversas condiciones, tanto en espacios interiores como exteriores.\\
\\
Si bien no se ha definido una aplicaci\'on espec\'ifica para el sistema dise\~nado, es claro que mediante modificaciones no demasiado grandes el mismo puede ser adaptado a m\'ultiples usos.\\
\\
El desarrollo del proyecto implicar\'a, entonces:

\begin{itemize}
\item Analizar el dispositivo comercial con el fin de comprender el sistema a controlar
\item Analizar el comportamiento f\'isico del sistema y modelar el mismo en forma adecuada
\item Realizar ingenier\'ia inversa al cuadric\'optero, su electr\'onica y su sistema de control. En particular, se prestar\'a especial detalle a los protocolos utilizados para el control de motores.
\item Poner a punto la instrumentaci\'on necesaria
\item Dise\~nar el sistema de procesamiento de datos de los sensores
\item Implementar las vias de comunicaci\'on necesarias para obtener datos a partir de los sensores, realizar el comando remoto del sistema, dar instrucciones e indicar la ruta a seguir al mismo, etc.
\item Dise\~nar e integrar el sistema de control
\item Lograr que el cuadricoptero sea capaz de realizar el seguimiento de una ruta en forma aut\'onoma
\end{itemize}

\end{document}