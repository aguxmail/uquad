\documentclass[main]{subfiles}

\begin{document}

\chapter{Especificaciones t\'ecnicas de la Mongoose 9DoF IMU}
\label{chap:anexo_mongoose}
\section{Aceler\'ometro}

\begin{table}[H]
\begin{center}
\rowcolors{1}{gray!20}{}
\begin{tabular}{|p{3cm}|p{6.5cm}|}
\hline
Rango & $\pm$16g \\
\hline
Resoluci\'on & 10-13 bits (siempre 4mg/LSB) \\
\hline
Datos nuevos &  0.1 a 800 Hz\\
\hline
Ru\'ido XY & 0.75@100Hz - 3@3200Hz LSB-rms\\
\hline
Ru\'ido Z & 1.1@100Hz - 4.5@3200Hz LSB-rms\\
\hline
Cross Axis & $\pm$ 1\% \\
\hline
\end{tabular}
\label{tab:acc-anexo}
\end{center}
\end{table}

\textbf{NOTAS}:
\begin{itemize}
\item Output data rate puede llegar a 3200Hz, pero usando SPI. Con I$^2$C a 400kHz solamente se puede llegar a 800Hz.
\item Ancho de banda = $Datos\_nuevos/2$
\end{itemize}

\section{Gir\'oscopo}

\begin{table}[H]
\begin{center}
\rowcolors{1}{gray!20}{}
\begin{tabular}{|p{3cm}|p{6.5cm}|}
\hline
Rango & $\pm$2000$^\circ/s$ \\
\hline
Resoluci\'on & 14.475 LSB/($^\circ/s$) \\
\hline
Datos nuevos &  3.9Hz a 8kHz\\
\hline
Ancho de banda & 256Hz \\
\hline
Cross Axis & $\pm$ 2\% \\
\hline
Ru\'ido & 0.38 $^\circ/s$-rms \\
\hline
\end{tabular}
\label{tab:gyro}
\end{center}
\end{table}

\textbf{NOTAS}:
\begin{itemize}
\item Datos nuevos: La muestras pasan por un LPF digital de 256 a 5Hz, esto limita el ancho de banda.
\end{itemize}

\section{Magnet\'ometro}

\begin{table}[H]
\begin{center}
\rowcolors{1}{gray!20}{}
\begin{tabular}{|p{3cm}|p{6.5cm}|}
\hline
Rango & $\pm$8 Ga\\
\hline
Resoluci\'on &  5mGa@GN=2\\
\hline
Datos nuevos &  0.75 - 75Hz\\
\hline
Ancho de banda &  37Hz\\
\hline
Cross Axis & $\pm$0.2\% FS/Ga \\
\hline
Ru\'ido & - \\
\hline
\end{tabular}
\label{tab:magn}
\end{center}
\end{table}

\textbf{NOTAS}:
\begin{itemize}
\item El rango queda determinado por la ganancia, que se configura con 3 bits:
\begin{table}[H]
\begin{center}
\rowcolors{1}{gray!20}{}
\begin{tabular}{|c|c|c|c|c|c|c|c|c|}
\hline
\textbf{GN} & 0 & 1 & 2 & 3 & 4 & 5 & 6 & 7 \\
\textbf{Rango} (Ga)& $\pm$0.88 & $\pm$1.3 & $\pm$1.9 & $\pm$2.5 & $\pm$4.0 & $\pm$4.7 & $\pm$5.6 & $\pm$8.1 \\
\hline
\end{tabular}
\label{tab:magn-gain}
\end{center}
\end{table}
\item Se puede configurar para que el dato que muestre sea el promedio de hasta 8 muestras.
\end{itemize}

\section{Sensor de presi\'on}

\begin{table}[H]
\begin{center}
\rowcolors{1}{gray!20}{}
\begin{tabular}{|p{3cm}|p{6.5cm}|}
\hline
Rango & 300 a 1100 hPa (9000 a -500m)\\
\hline
Resoluci\'on &  1Pa\\
\hline
Precisi\'on. Abs. & typ/max $\pm$1.0/$\pm$3.0 hPa \\
\hline
Precisi\'on Rel. & $\pm$0.5 hPa \\
\hline
Datos nuevos &  typ/max: 3/4.5ms - 17/25ms\\
\hline
Ancho de banda &  333/40Hz\\
\hline
Ru\'ido (hPa) &  0.06 - 0.03\\
\hline
Ru\'ido (m) & 0.5 - 0.25 \\
\hline
\end{tabular}
\label{tab:barometro}
\end{center}
\end{table}

\textbf{NOTAS}:
\begin{itemize}
\item El rango, en altura, se refiere a la altura sobre el nivel del mar.
\item El modo de operaci\'on (cantidad de muestras promediadas) afecta:
  \begin{itemize}
  \item El tiempo de conversi\'on.
  \item El ancho de banda.
  \item La resoluci\'on.
  \item El ru\'ido.
  \end{itemize}
\item Es necesario hacer una medida de temperatura de vez en cuando (1Hz) para mejorar la lectura del sensor de presi\'on.
\end{itemize}

\section{Sensor de temperatura}

\begin{table}[H]
\begin{center}
\rowcolors{1}{gray!20}{}
\begin{tabular}{|p{3cm}|p{6.5cm}|}
\hline
Rango & 0 a 65 $^\circ$C\\
\hline
Resoluci\'on &  0.1 $^\circ$C\\
\hline
Precisi\'on Abs. & typ/max $\pm$1.0/$\pm$2.0 $^\circ$C\\
\hline
Datos nuevos &  typ/max: 3/4.5ms\\
\hline
Ru\'ido & - \\
\hline
\end{tabular}
\label{tab:temp}
\end{center}
\end{table}

\textbf{NOTAS}:
\begin{itemize}
\item El sensor de temperatura est\'a incorporado al sensor de presi\'on.
\end{itemize}

\end{document}