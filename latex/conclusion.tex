\documentclass[main]{subfiles}
\begin{document}

\chapter*{Conclusi\'on}

A lo largo del presente texto se ha intentado transmitir la esencia  del trabajo realizado durante la totalidad del Proyecto de Fin de Carrera. En lo que respecta estrictamente a los objetivos trazados se lograron cumplir la mayoría. Sin embargo el proyecto result\'o ambicioso en cuanto al alcance planteado inicialmente en funci\'on de la poca experiencia del grupo en la gran mayor\'ia de los desaf\'ios que se encontraron. El proyecto en el que se trabaj\'o comprende una gran variedad de \'areas como la electr\'onica y el procesamiento de señales, pero es en esencia, un proyecto de control. Parad\'ojicamente, los problemas espec\'ificos sobre el diseño en s\'i del controlador y del modelado del sistema no fueron las \'areas en las que se encontraron mayores dificultades, probablemente debido a la experiencia ya adquirida a lo largo de la formaci\'on que hemos recibido en la Facultad de Ingeniería de la Universidad de la República. Previo al comienzo del proyecto, las dificultades de pasar de la teor\'ia y de las simulaciones al sistema real eran inimaginables. Esta fue probablemente la mayor dificultad frente a la cual nos enfrentamos diariamente a lo largo de este trabajo.\\

A pesar de las dificultades planteadas y de no lograr cumplir con la totalidad de los ambiciosos objetivos trazados inicialmente, se desprenden diversas valoraciones positivas en cuanto a las enseñanzas del proyecto tanto en lo acad\'emico como en lo que refiere a la din\'amica del trabajo en grupo.\\

En lo que respecta al alcance del proyecto, la gran mayoría de los puntos planteados fueron logrados con \'exito. Se modeló el sistema, se desarrolló un simulador donde es posible testear la performance de las técnicas de control implementadas, se logró la integración del sistema dejando una plataforma estable y con gran capacidad, se decodificó el protocolo de comunicación con los motores logrando actuar sobre ellos, se desarrolló un control LQR automático y fundamentalmente \textbf{se logró el vuelo autónomo de una plataforma aérea no tripulada}. A su vez, como objetivo secundario, se desarrolló una aplicación para obtener la posición y orientación del cuadricóptero con una muy buena precisión (del orden de algunos centímetros en la posición), utilizando una cámara montada en el cuadricóptero, la cual por razones de tiempo no pudo ser testeada sobre el mismo (por más detalles, visitar el anexo \ref{chap:camara}).\\

Adem\'as del modelado del sistema se desarroll\'o la teor\'ia que permite trabajar con algunos tipos de trayectorias como rectas, c\'irculos y equilibrio logrando simulaciones que arrojan resultados muy satisfactorios. Lamentablemente, por falta de tiempo estas trayectorias no pudieron ser testeadas en la pr\'actica completamente. La \'unica prueba que se realiz\'o fue la de lograr el \emph{hovering}, logrando resultados ampliamente satisfactorios.\\

Por otro lado, y aún más importante, se implementaron dos plataformas id\'enticas que le permiten al departamento de Sistemas y Control del instituto de Inegenier\'ia El\'ectrica de la Facultad de Ingenier\'ia contar con nuevos elementos para continuar desarrollando t\'ecnicas de control orientas a la navegaci\'on no tripulada, ya sea en lineas de investigaci\'on independientes, en futuros Proyectos de Fin de Carrera o postgrados. El sistema integrado permite, con algunas salvedades, la puesta a prueba de las t\'ecnicas a desarrollar en un ambiente de laboratorio.\\

Gran parte del trabajo realizado consiti\'o en comprender la forma en que se deb\'ia actuar sobre los motores (dado que esta informaci\'on no se encontraba disponible), y en comprender como lograr una adecuada medida de las variables de estado en funci\'on de los sensores de los que se dispon\'ia. La falta de experiencia llev\'o adem\'as a no realizar las compras adecuadas en una primera instancia, teniendo as\'i demoras a lo largo del proyecto y duplicaci\'on del trabajo. A modo de ejemplo fue necesaria la inclusión de un magnetómetro, barómetro y termómetro para poder lograr una medida completa de todas las variables de interés, sensores que no fueron previstos en una primera instancia. Más allá de los contratiempos, idas y vueltas, quedó demostrado que efectivamente es posible lograr el vuelo autónomo de una vehículo aéreo no tripulado con los sensores utilizados, presupuesto acotado y mucha dedicación.\\

%Entre los objetivos que no pudieron realizarse se encuentra la implementaci\'on del generador de rutas. De todas formas, este elemento no resulta fundamental en lo que respecta al vuelo del cuadric\'optero ya que las restricciones f\'isicas del sistema son muy pocas a diferencia por ejemplo de un avi\'on, el cual no puede realizar giros sin que estos tengan un radio m\'inimo.\\

Como se adelant\'o en la introducci\'on a este texto no era un objetivo lograr un producto final ni definitivo, sino que se trataba de integrar una plataforma que permita continuar la investigaci\'on en tem\'aticas relacionadas con la navegaci\'on. En particular a partir de este momento se abren diversas lineas de investigaci\'on. En lo que refiere por ejemplo a la estimaci\'on del estado la principal linea a explorar es la inclusi\'on de cam\'aras ya sea en el cuadric\'optero o en la zona de prueba, a fin de evitar el problema de la falta de GPS en ambientes cerrados. Otra línea de trabajo posible en lo que refiere a la estimaci\'on del estado consiste en explorar en profundidad la forma de reducir los errores aportados por los sensores de forma de lograr mejores estimaciones de algunas variables (en particular las velocidades y posiciones en el plano horizontal).\\

A lo largo del presente proyecto se observaron en diversas oportunidades deficiencias en el control de los motores, en particular se comprob\'o que los mismos tienen un grave problema de diseño en lo que respecta a la disipaci\'on t\'ermica, produciendo que algunos de los transistores de potencia que los componen se quemen. Esto produce que algunos motores dejen de funcionar. Esta deficiencia se observ\'o sobre el final del proyecto en el momento de realizar las pruebas sobre el sistema completo, por lo tanto no hubo tiempo de resolver dicha dificultad. Debe se\~nalarse la necesidad de re-diseñar estos controladores o adquirir otros de forma de eliminar un factor de riesgo. As\'imismo puede aprovecharse para mejorar otros aspectos secundarios de los ESCs como por ejemplo lograr eliminar la variaci\'on de la velocidad que los motores presentan en funci\'on de la bater\'ia disponible. As\'imismo, para algunas aplicaciones puede ser interesante contar con motores que puedan girar tanto en sentido horario como anti-horario.\\

Finalmente se puede nombrar una linea de trabajo orientada a la creaci\'on de rutas \'optimas y el desarrollo de algor\'itmos de control m\'as sofisticados que involucren t\'ecnicas de control no lineales de forma de abrir el abanico de las trayectorias realizables. A modo de ejemplo a lo largo de este proyecto trabajamos con giros respecto del eje vertical, sin embargo algunas maniobras requieren giros respecto de otros ejes, pero las t\'ecnicas de control desarrolladas en este proyecto limitan estas posibilidades.\\

El trabajo realizado a lo largo del Proyecto de Fin de Carrera nos permiti\'o enfrentarnos a un problema de ingenier\'ia real y suficientemente complejo de el cual se obtienen enseñanzas m\'as all\'a del plano acad\'emico. Principalmente se debe destacar la forma de encarar un problema que tiene una gran cantidad de sub problemas asociados que deben resolverse para lograr el correcto funcionamiento del conjunto. La identificaci\'on de las diferentes problem\'aticas a resolver es un ejercicio que hasta el momento no se hab\'ia realizado. El segundo aspecto a destacar es la autogesti\'on del tiempo para cumplir los objetivos a largo plazo, los plazos del mismo (a pesar de las entregas intermedias) son impuestos fundamentalmente por el grupo a diferencia de la experiencia en el resto de las asignaturas de la facultad. 

\end{document}