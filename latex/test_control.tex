\documentclass[main]{subfiles}

\begin{document}


\chapter{Pruebas del controlador}
El controlador diseñado se comporta adecuadamente en lo que respecta a las simulaciones, sin embargo debido a que la caracterizaci\'on del sistema puede contener errores se proceden a realizar algunas pruebas sobre los subsistemas que componen al cuadric\'optero. Estas pruebas son de utilidad para verificar el correcto funcionamiento del controlador diseñado y/o para realizar los ajustes que sean necesarios en el mismo.\\

\section{Control del subsistema del Roll}

Para lograr el correcto funcionamiento del cuadric\'optero es fundamental que el control sobre los \'angulos de Pitch y de Roll se comporte de buena forma. Estos \'angulos son claves, a modo de ejemplo, es imposible lograr el equilibrio mec\'anico si dichos \'angulos difieren de cero. Por dicha raz\'on previo a realizar pruebas sobre el sistema completo es necesario asegurarnos que los subsistemas del  Roll y del Pitch funcionan correctamente. De acuerdo al modelo f\'isico del sistema desarrollado en \ref{chap:modelo_fisico} ni el Roll ni el Pitch son subsistemas independientes entre s\'i, adem\'as ambos dependen de la velocidad angular seg\'un $\vec{k}_q$. Sin embargo, dichos \'angulos toman valores cercanos a cero en las trayectorias de inter\'es, en este caso se puede realizar la aproximaci\'on de que ambos sistemas son independientes.\\

A partir de esta consideraci\'on se procede a fijar al cuadric\'optero sobre dos gu\'ias como se muestra en la figurea de forma de eliminar todos los grados de libertad del sistema excepto el \'angulo de Roll y la velocidad angular correspondiente al eje de rotaci\'on del Roll. Se realizan dos pruebas: la primera consiste en que el sistema alcance la posic\'on de equilibrio ($\psi = 0$), la segunda consiste en alejar al sistema del equilibrio y lograr que vuelva al punto de equilibrio.

\end{document}