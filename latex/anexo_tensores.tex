\documentclass[main]{subfiles}
\begin{document}

\appendix

\cleardoublepage
\addappheadtotoc
\appendixpage
\renewcommand{\appendixname}{Anexo}
\renewcommand{\appendixtocname}{Anexo}
\renewcommand{\appendixpagename}{Anexos}
\chapter{C�lculo de los tensores}


\section{Magnitudes a considerar}
Seg�n el modelo considerado explicado en \ref{chap:modelo} dividiremos el sistema en una esfera central, cuatro cilindros y cuatro varillas (tambi�n cil�ndricas). Las magnitudes que debemos conocer para calcular los momentos de inercia del sistema son:

\begin{itemize}
\item Radio de la esfera central $R=8\times10^{-2}m$
\item Largo de las varillas $L=26\times10^{-2}m$
\item Radio de los motores $r=1.65\times10^{-2}m$
\item Altura de los motores $h=3.5\times10^{-2}m$
\end{itemize}

Adem�s nos interesa conocer las distancias de cada uno de los elementos del sistema al centro de masa del cuadic�ptero.

\begin{itemize}
\item Distancia entre el centro de masa de la varilla y el centro de masa del cuadric�ptero $d_v=14\times10^{-2}m$
\item Distancia del eje de los motores al centro de masa del cuadric�ptero $d_m=0.29m$
\end{itemize}

Por �ltimo las masas de los elementos en cuesti�n son:

\begin{itemize}
\item Masa de la esfera central $M_E=1.037kg$
\item Masa de las varillas $M_v=0.013kg$
\item Masa de los motores $M_m=0.113kg$
\end{itemize}



\section{Tensor de inercia del sistema}

\label{tensores}

El tensor de inercia del sistema puede calcularse como la suma de los tensores de inercia de los r�gidos que lo componen. Se considera como fue expresado anteriormente el centro del cuadric�ptero como una esfera maciza. El tensor de inercia de dicha esfera puede calcularse a partir de la definici�n misma de tensor de inercia, sin embargo por ser una forma geom�trica de vasto uso en el campo de la mec�nica su tensor de inercia se encuentra ya tabulado. Sucede lo mismo con las restantes formas geom�tricas que componen al sistema. Los tensores utilizados pueden obtenerse en \ref{bib:tensores} En el caso de la esfera se tiene que el tensor de inercia respecto de  su centro de masa es: 

$$
\Pi_{G_E}^{\{\vec{i_q}, \vec{j_q}, \vec{k_q}\}}= M_E\left(\begin{array}{ccc}
\frac{2R^2}{5}  &0&  0\\
0  &\frac{2R^2}{5} & 0\\
0  &0 & \frac{2R^2}{5} \\
\end{array}\right) \\
$$

En este caso el centro de masa del sistema corresponde al centro de masa de la esfera a partir de ciertas suposiciones que se realizan sobre la simetr�a del sistema. Por dicho motivo podemos afirmar que $\Pi_{G_E}^{\{\vec{i_q}, \vec{j_q}, \vec{k_q}\}} = \Pi_{O\prime _E}^{\{\vec{i_q}, \vec{j_q}, \vec{k_q}\}} $, siendo $O\prime$ el centro de la esfera.

Por otra parte el tensor de inercia de una varilla, cuya longitud coincide con el versor $\vec{i_q}$, respecto a su centro de masa tiene la forma:

$$
\Pi_{G_{Vx}}^{\{\vec{i_q}, \vec{j_q}, \vec{k_q}\}}= M_V\left(\begin{array}{ccc}
0  &0&  0\\
0  &\frac{L^2}{12} & 0\\
0  &0 & \frac{L^2}{12}  \\
\end{array}\right) \\ 
$$

Sin embargo resulta mucho m�s interesante obtener el tensor de inercia expresado respecto del centro de masa del sistema. Para realizar dicho cambio se utiliza el Teorema de Steiner. Dicho teorema afirma que: $\Pi_Q = \Pi_G +J_Q^{M,G}$, donde los t�rminos de $J_Q^{M,G}$ pueden calcularse como: $(J_Q^{M,G})_{\alpha \beta} = M(G-Q)^2\delta_{\alpha \beta}-M(G-Q)_{\alpha}M(G-Q)_{\beta}$. El t�rmino $\delta_{\alpha \beta}$ es conocido como Delta de Kronecker. Su valor es uno si $\alpha =\beta$ y cero si $\alpha \neq \beta$. En el caso en consideraci�n dicha matriz resulta en:

$$
J_O\prime^{M_{Vx},G}= M_V \left(\begin{array}{ccc}
0  &0&  0\\
0  & (\frac{L}{2}+d_v)^2 & 0\\
0  &0 & (\frac{L}{2}+d_v)^2  \\
\end{array}\right) \\
$$

Por lo tanto el momento de inercia total de dicha varilla es:
$$
\Pi_{O\prime_{Vx}}^{\{\vec{i_q}, \vec{j_q}, \vec{k_q}\}}=M_V \left(\begin{array}{ccc}
0  &0&  0\\
0  &\frac{L^2}{3}+(\frac{Ld_v}{2})^2+d_v^2 & 0\\
0  &0 & \frac{L^2}{3}+(\frac{Ld_v}{2})^2+d_v^2  \\
\end{array}\right)$$


An�logamente, el tensor de inercia de una varilla cuya longitud se encuentra respecto de la direcci�n $\vec{j_q}$ respecto del centro de masa del sistema es:

$$
\Pi_{O\prime_{Vy}}^{\{\vec{i_q}, \vec{j_q}, \vec{k_q}\}}=M_V \left(\begin{array}{ccc}
\frac{L^2}{3}+(\frac{Ld_v}{2})^2+d_v^2  &0 & 0\\
0  &0 & 0\\
0  &0 & \frac{L^2}{3}+(\frac{Ld_v}{2})^2+d_v^2  \\
\end{array}\right)$$


Sucede algo similar en lo que respecta a los motores. Tendremos un tensor de inercia para los motores que se encuentran sobre la direcci�n $\vec{i_q}$ y otro para los motores que se encuentran sobre la direcci�n $\vec{j_q}$. El momento de inercia de un cilindro en su centro de masa es:

\begin{equation}
\label{eq:izzm}
\Pi_{G_{M}}^{\{\vec{i_q}, \vec{j_q}, \vec{k_q}\}}=M_M\left(\begin{array}{ccc}
\frac{3r^2+h^2}{12}  &0&  0\\
0  &\frac{3r^2+h^2}{12} & 0\\
0  &0 & \frac{r^2}{2}  
\end{array}\right)
\end{equation}
La entrada $(3,3)$ de la matriz de la ecuaci�n \ref{eq:izzm} la pasremos a llamar $I_{zzm}$\\

Por lo tanto el momento de inercia respecto del centro de masa del sistema 1para un cilindro que se encuentra en la direcci�n $\vec{i_q}$ es:
$$
\Pi_{O\prime_{Mx}}^{\{\vec{i_q}, \vec{j_q}, \vec{k_q}\}}=M_M\left(\begin{array}{ccc}
\frac{3r^2+h^2}{12}  &0&  0\\
0  &\frac{3r^2+h^2}{12} +d_m^2& 0\\
0  &0 & \frac{r^2}{2}+d_m^2  
\end{array}\right)
$$


En el caso de un cilindro que se encuentra en la direcci�n $\vec{j_q}$ se tiene que::
$$
\Pi_{O\prime_{My}}^{\{\vec{i_q}, \vec{j_q}, \vec{k_q}\}}=M_M\left(\begin{array}{ccc}
\frac{3r^2+h^2}{12} +d_m^2 &0&  0\\
0  &\frac{3r^2+h^2}{12} & 0\\
0  &0 & \frac{r^2}{2}+d_m^2  
\end{array}\right)
$$


Finalmente, el tensor de inercia del sistema, se calcula como la suma de los tensores de inercia de las partes que lo componen:
$$\Pi_{O\prime}^{\{\vec{i_q}, \vec{j_q}, \vec{k_q}\}} =\Pi_{O\prime_E}^{\{\vec{i_q}, \vec{j_q}, \vec{k_q}\}} + 2 \Pi_{O\prime_{Vx}}^{\{\vec{i_q}, \vec{j_q}, \vec{k_q}\}} + 2 \Pi_{O\prime_{Vy}}^{\{\vec{i_q}, \vec{j_q}, \vec{k_q}\}} + 2 \Pi_{O\prime_{Mx}}^{\{\vec{i_q}, \vec{j_q}, \vec{k_q}\}}+ 2 \Pi_{O\prime_{My}}^{\{\vec{i_q}, \vec{j_q}, \vec{k_q}\}}$$.

Dado que todos los tensores de inercia considerados hasta el momento son diagonales, podemos escribir el tensor de inercia del sistema como:

$$\Pi_{O\prime}^{\{\vec{i_q}, \vec{j_q}, \vec{k_q}\}}=\left(\begin{array}{ccc}
I_{xx}  &0&  0\\
0  &I_{yy}& 0\\
0  &0 & I_{zz}  
\end{array}\right)$$

\section{Resultados} 

En base al an�lisis realizado hasta el momento se tiene que: 

$$I_{zmm}=1.54\times10^{-5}kgm^2
$$

$$
I_{xx}=I_{yy}=2.32\times10^{-2}kgm^2
$$

$$
I_{zz}=4.37\times10^{-2}kgm^2
$$
\end{document}
