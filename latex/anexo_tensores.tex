 	\documentclass[main]{subfiles}
\begin{document}

\appendix

\cleardoublepage
\addappheadtotoc
\appendixpage
\renewcommand{\appendixname}{Anexo}
\renewcommand{\appendixtocname}{Anexo}
\renewcommand{\appendixpagename}{Anexos}
\chapter{C�lculo de los tensores}


Bajo estas suposiciones las cantidades de inter�s en lo que refiere a las dimensiones de sistema son:

\begin{itemize}
\item Radio de la esfera central (R)
\item Largo de las varillas (L)
\item Radio de los motores(r)
\end{itemize}


\subsection{Masa del sistema}
La masa de los objetos que componen al sistema son:

\begin{table}[H]
\centering
\begin{tabular}{p{80pt}|p{80pt}|p{80pt}|p{80pt}|} 
\cline{2-3}
& \cellcolor[gray]{0.8} \textbf{Masa por elemento} 
& \cellcolor[gray]{0.8} \textbf{Cantidad} 
& \cellcolor[gray]{0.8} \textbf{Masa total} \\ \cline{2-4} \hline
\multicolumn{1}{|p{80pt}|}{\cellcolor[gray]{0.8}\textbf{Esfera Central}} 
& $M_E$  & 1 & $M_E$ \\ \hline
\multicolumn{1}{|p{80pt}|}{\cellcolor[gray]{0.8}\textbf{Varilla}} 
& $M_V$ & 4 &$4 M_V$ \\ \hline
\multicolumn{1}{|p{80pt}|}{\cellcolor[gray]{0.8}\textbf{Motores}} 
&$M_M$  & 4 & $4 M_M$\\ \hline
\multicolumn{1}{|p{80pt}|}{\cellcolor[gray]{0.8}\textbf{Masa Total}} & M\\
\cline{1-2}
\end{tabular}
\caption{Masas de los objetos que componen al sistema}
\label{tab:masas}
\end{table}



\subsection{Tensor de inercia del sistema}

\label{tensores}

El tensor de inercia del sistema puede calcularse como la suma de los tensores de inercia de los r�gidos que lo componen. Se considera como fue expresado anteriormente el centro del cuadric�ptero como una esfera maciza. El tensor de inercia de dicha esfera puede calcularse a partir de la definici�n misma de tensor de inercia, sin embargo por ser una forma geom�trica de vasto uso en el campo de la mec�nica su tensor de inercia se encuentra ya tabulado. Sucede lo mismo con las restantes formas geom�tricas que componen al sistema. En el caso de la esfera se tiene que el tensor de inercia respecto de  su centro de masa es: 

$$
\Pi_{G_E}^{\{\vec{i_q}, \vec{j_q}, \vec{k_q}\}}= M_E\left(\begin{array}{ccc}
\frac{2R^2}{5}  &0&  0\\
0  &\frac{2R^2}{5} & 0\\
0  &0 & \frac{2R^2}{5} \\
\end{array}\right) \\
$$

En este caso el centro de masa del sistema corresponde al centro de masa de la esfera a partir de ciertas suposiciones que se realizan sobre la simetr�a del sistema. Por dicho motivo podemos afirmar que $\Pi_{G_E}^{\{\vec{i_q}, \vec{j_q}, \vec{k_q}\}} = \Pi_{O\prime _E}^{\{\vec{i_q}, \vec{j_q}, \vec{k_q}\}} $, siendo $O\prime$ el centro de la esfera.
%A partir de las dimensiones y masas expresadas en las secciones anteriores se deduce que:\\

%$\Pi_{O\prime_E}^{\{\vec{i_q}, \vec{j_q}, \vec{k_q}\}}}= \left(\begin{array}{ccc}%

%  &0&  0\\
%0  & & 0\\
%0  &0 &  \\

%\end{array}\right) $\\
Por otra parte el tensor de inercia de una varilla, cuya longitud coincide con el versor $\vec{i_q}$, respecto a su centro de masa tiene la forma:

$$
\Pi_{G_{Vx}}^{\{\vec{i_q}, \vec{j_q}, \vec{k_q}\}}= M_V\left(\begin{array}{ccc}
\frac{L^2}{24}  &0&  0\\
0  &\frac{L^2}{12} & 0\\
0  &0 & \frac{L^2}{12}  \\
\end{array}\right) \\ 
$$

Sin embargo resulta mucho m�s interesante obtener el tensor de inercia expresado respecto del centro de masa del sistema. Para realizar dicho cambio se utiliza el Teorema de Steiner. Dicho teorema afirma que: $\Pi_Q = \Pi_G +J_Q^{M,G}$, donde los t�rminos de $J_Q^{M,G}$ pueden calcularse como: $(J_Q^{M,G})_{\alpha \beta} = M(G-Q)^2\delta_{\alpha \beta}-M(G-Q)_{\alpha}M(G-Q)_{\beta}$. El t�rmino $\delta_{\alpha \beta}$ es conocido como Delta de Kronecker. Su valor es uno si $\alpha =\beta$ y cero si $\alpha \neq \beta$. En el caso en consideraci�n dicha matriz resulta en:

$$
J_O\prime^{M_{Vx},G}= M_V \left(\begin{array}{ccc}
0  &0&  0\\
0  & (\frac{L}{2}+R)^2 & 0\\
0  &0 & (\frac{L}{2}+R)^2  \\
\end{array}\right) \\
$$

Por lo tanto el momento de inercia total de dicha varilla es:
$$
\Pi_{O\prime_{Vx}}^{\{\vec{i_q}, \vec{j_q}, \vec{k_q}\}}=M_V \left(\begin{array}{ccc}
\frac{L^2}{24}  &0&  0\\
0  &\frac{L^2}{3}+\frac{LR}{2}^2+R^2 & 0\\
0  &0 & \frac{L^2}{3}+\frac{LR}{2}^2+R^2  \\
\end{array}\right)$$
%=\left(\begin{array}{ccc}
%\frac{L^2}{6}  &0&  0\\
%0  &\frac{L^2}{12}+(\frac{L}{2}+R)^2 & 0\\
%0  &0 & \frac{L^2}{12}+(\frac{L}{2}+R)^2  \\
%\end{array}\right)\\


An�logamente, el tensor de inercia de una varilla cuya longitud se encuentra respecto de la direcci�n $\vec{j_q}$ respecto del centro de masa del sistema es:

$$
\Pi_{O\prime_{Vy}}^{\{\vec{i_q}, \vec{j_q}, \vec{k_q}\}}=M_V \left(\begin{array}{ccc}
\frac{L^2}{3}+\frac{LR}{2}^2+R^2  &0&  0\\
0  &\frac{L^2}{24} & 0\\
0  &0 & \frac{L^2}{3}+\frac{LR}{2}^2+R^2  \\
\end{array}\right)$$
%= \left(\begin{array}{ccc}
%\frac{L^2}{6}  &0&  0\\
%0  &\frac{L^2}{12}+(\frac{L}{2}+R)^2 & 0\\
%0  &0 & \frac{L^2}{12}+(\frac{L}{2}+R)^2  
%\end{array}\right)


Sucede algo similar en lo que respecta a los motores. Tendremos un tensor de inercia para los motores que se encuentran sobre la direcci�n $\vec{i_q}$ y otro para los motores que se encuentran sobre la direcci�n $\vec{j_q}$. Aproximando cada motor por un cilindro obtenemos en el primer caso el tensor de inercia tiene el valor:

$$
\Pi_{O\prime_{Mx}}^{\{\vec{i_q}, \vec{j_q}, \vec{k_q}\}}=M_M\left(\begin{array}{ccc}
\frac{r^2}{4}  &0&  0\\
0  &\frac{r^2}{4}+(L+R+r)^2 & 0\\
0  &0 & \frac{r^2}{2}+(L+R+r)^2  
\end{array}\right)
$$

En el otro caso se tiene que:
$$\Pi_{O\prime_{My}}^{\{\vec{i_q}, \vec{j_q}, \vec{k_q}\}}=M_M\left(\begin{array}{ccc}
\frac{r^2}{4}+(L+R+r)^2  &0&  0\\
0  &\frac{r^2}{4} & 0\\
0  &0 & \frac{r^2}{2}+(L+R+r)^2  \\
\end{array}\right)$$
Llamaremos por conveniencia $I_{zz_m}$ al termino $\frac{r^2}{2}+(L+R+r)^2$. El tensor de inercia del sistema completo puede calcularse como:
$$\Pi_{O\prime}^{\{\vec{i_q}, \vec{j_q}, \vec{k_q}\}} =\Pi_{O\prime_E}^{\{\vec{i_q}, \vec{j_q}, \vec{k_q}\}} + 2 \Pi_{O\prime_{Vx}}^{\{\vec{i_q}, \vec{j_q}, \vec{k_q}\}} + 2 \Pi_{O\prime_{Vy}}^{\{\vec{i_q}, \vec{j_q}, \vec{k_q}\}} + 2 \Pi_{O\prime_{Mx}}^{\{\vec{i_q}, \vec{j_q}, \vec{k_q}\}}+ 2 \Pi_{O\prime_{My}}^{\{\vec{i_q}, \vec{j_q}, \vec{k_q}\}}$$.

Podemos escribir dicho tensor de inercia como:

$$\Pi_{O\prime_{My}}^{\{\vec{i_q}, \vec{j_q}, \vec{k_q}\}}=\left(\begin{array}{ccc}
I_{xx}  &0&  0\\
0  &I_{yy}& 0\\
0  &0 & I_{zz}  
\end{array}\right)$$

%Por lo tanto tenemos:\\

%$\Pi_{O\prime}^{\{\vec{i_q}, \vec{j_q}, \vec{k_q}\}}} =$


\end{document}
