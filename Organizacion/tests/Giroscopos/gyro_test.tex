\documentclass[spanish,12pt,a4paper,titlepage]{report}
\usepackage[utf8]{inputenc}
\usepackage{graphicx}
\usepackage{subfig}
\usepackage{float}
\usepackage{wrapfig}
\usepackage{multirow}
\usepackage{caption}
\usepackage[spanish]{babel}
\usepackage[dvips]{hyperref}
\usepackage{amssymb}
\usepackage{listings}
\usepackage{epsfig}
\usepackage{amsmath}
\usepackage{array}
\usepackage[table]{xcolor}
\usepackage{multirow}
%\usepackage[Sonny]{fncychap}
\usepackage[Lenny]{fncychap}
%\usepackage[Glenn]{fncychap}
%\usepackage[Conny]{fncychap}
%\usepackage[Rejne]{fncychap}
%\usepackage[Bjarne]{fncychap}
%\usepackage[Bjornstrup]{fncychap}

%\usepackage{subfiles}
%\usepackage{framed}

\setlength{\topmargin}{-1.5cm}
\setlength{\textheight}{25cm}
\setlength{\oddsidemargin}{0.3cm} 
\setlength{\textwidth}{15cm}
\setlength{\columnsep}{0cm}

\begin{document}

\chapter{Calibración de giróscopo}

\section{Objetivos}

Se realiza una serie de pruebas con el fin de calibrar el acelerómetro de tres ejes de la IMU. Se estudiarán diversos modelos posibles para dicha calibración (lineal, cúbico,etc). Se estimaran además las fuentes de error del mismo (ruido y bias).

\section{Materiales}
\begin{itemize}
\item IMU
\item Beagleboard
\item Tocadiscos
\item Cronómetro
\item Regla
\end{itemize}
\section{Procedimiento}

\subsection{Parte I: Calibración}

\subsubsection*{Medidas a Velocidad angular constante}
En primer lugar se debe obtener la velocidad angular del tocadiscos. Se toma el tiempo que demora el tocadiscos en dar 10 vueltas completas. Se ubica el sistema exactamente en el centro del disco con el eje \textbf{z} perpendicular al mismo. Se toman medidas durante un minuto con una frecuencia de 100 muestras por segundo. Se repite el experimento una vez con el eje \textbf{y} perpendicular al disco y otra con el eje \textbf{x} perpendicular al disco. Se contrasta la velocidad angular en las 3 direcciones con la velocidad de giro del tocadiscos.

\subsection{Parte II: Definición del modelo}
A partir de las medidas obtenidas y los valores teóricos de las velocidades angulares en cada experimento realizado se propone una aproximación lineal y una cúbica. Se obtienen los parámetros de dichos modelos con el método de mínimos cuadrados.\\

Se ajusta la curva según los modelos propuestos. Se opta por el modelo que de una mejor respuesta.

\subsection{Parte III: Determinación de no idealidades}
A esta altura se intenta caracterizar las dos fuentes principales de no idealidades de los acelerómetros: El ruido y el Bias.

\subsubsection*{Ruido}
Se modela el ruido propio del giróscopo como gaussiano de media nula. Se toman medidas en reposo en los tres ejes durante 10 minutos. A partir de dichas medidas se estima la potencia del ruido. Se diseña un filtro para disminuir dicho ruido. Se grafican las aceleraciones de todos los experimentos realizados hasta el momento con el filtrado definido en este experimento. Se comparan las nuevas respuestas de los acelerómetros con las obtenidas antes del filtrado. 

\subsubsection*{Bias}
Se toman muestras de 10 minutos en reposo para los tres ejes. Se integran dichas muestras y se observa el error de velocidad y posición obtenidos. Se repite el experimento 2 veces más con algunas horas de diferencia. Se comparan los resultados obtenidos, se estudia si hay algún patrón.\\

Realizar la integración del error teórico en la posición, asegurado por el fabricante, según la hoja de datos. Comparar con el bias obtenido. Determinar si los sensores funcionan como lo asegura el fabricante.

\end{document}